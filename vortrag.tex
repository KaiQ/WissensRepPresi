% http://www2.informatik.hu-berlin.de/~mischulz/beamer.html
%
% Erzeugung: latex, latex, dvi2ps, viewps
%
%\documentclass[notes,xcolor=dvipsnames,14pt]{beamer}
\documentclass[xcolor=dvipsnames,12pt,aspectratio=43]{beamer}
\usepackage{etex}
\usenavigationsymbolstemplate{}
\usepackage{pgfpages}
%\pgfpagesuselayout{resize to}[a4paper,border shrink=5mm,landscape]
\usecolortheme[named=Gray]{structure}
% {Apricot | Aquamarine | Bittersweet | Black | Blue | BlueGreen | BlueViolet | BrickRed | Brown | BurntOrange | CadetBlue | CarnationPink | Cerulean | CornowerBlue | Cyan | Dandelion | DarkOrchid | Emerald | ForestGreen | Fuchsia | Goldenrod | Gray | Green | GreenYellow | JungleGreen | Lavender | LimeGreen | Magenta | Mahogany | Maroon | Melon | MidnightBlue | Mulberry | NavyBlue | OliveGreen | Orange | OrangeRed | Orchid | Peach | Periwinkle | PineGreen | Plum | ProcessBlue | Purple | RawSienna | Red | RedOrange | RedViolet | Rhodamine | aRoyalBlue | RoyalPurple | RubineRed | Salmon | SeaGreen | Sepia | SkyBlue | SpringGreen | Tan | TealBlue | Thistle | Turquoise | Violet | VioletRed | White | WildStrawberry | Yellow | YellowGreen | YellowOrange}
\usepackage{floatflt}
\usepackage{wrapfig}
%\usepackage{beamerthemesplit} % kam neu dazu
\usepackage{eurosym}

%\usepackage[utf8x]{inputenc}
\usepackage[ngerman]{babel}
%\usepackage[latin1]{inputenc}
\usepackage[utf8]{inputenc}
\usepackage{colortbl}
\usepackage{color}
\usepackage{xcolor}
\usepackage{listings}
%-------------------------------------------------------------------
% Zeichnen der Diagramme
\usepackage{tikz}
\usepackage{pgfplots} % LaTeX
%\usemodule[pgfplots] % ConTeXt
\usetikzlibrary{patterns}
\usetikzlibrary{shapes,arrows,positioning}
%\usetikzlibrary{graphdrawing}
\usetikzlibrary{graphs}
\usetikzlibrary{trees}
\tikzstyle{selected}=[draw=red,fill=red!30]
\tikzstyle{optional}=[dashed,fill=gray!50]
\usepackage{verbatim}
\usepackage{amssymb}
\usepackage{ulem}
\usepackage{fancybox}

\tikzstyle{decision} = [diamond, draw, fill=gray!50,text width=4.5em,
text centered,node distance=3cm, inner sep=0pt]
%\tikzstyle{block} = [rectangle, draw, fill=gray!30,text width=5em, text centered, rounded corners, minimum height=4em]
\tikzstyle{block} = [rectangle, draw, fill=gray!30, text centered, rounded corners, minimum height=2.5em]
\tikzstyle{line} = [draw, -latex']
\tikzstyle{cloud} = [draw, ellipse,fill=gray!10,minimum height=4em]
\tikzstyle{io} =  [draw,trapezium,trapezium left angle=70,trapezium right angle=-70,minimum height=2.5em, fill=gray!70]
\tikzstyle{prozess} = [draw,rectangle split, rectangle split horizontal,rectangle split parts=3,minimum height=4em , node distance=3cm, fill=gray!90 ]


%-------------------------------------------------------------------Captions

% \usepackage[small, bf, format=hang]{caption}
\usepackage[babel,german=guillemets]{csquotes}
% \renewcommand{\captionfont}{\tiny}
% \captionsetup[figure]{labelfont=bf,textfont=it, labelfont={color=gray}}
% \captionsetup[table]{labelfont=bf,textfont=it, labelfont={color=gray}}
%------------------------------------------------------------------------------
%\usepackage{graphicx}
%\usepackage{epsfig}

\usepackage{graphicx}
%\usepackage{graphics}
%\usepackage{epsf}
%\usepackage{epsfig}

\definecolor{violett}{rgb}{0.800,0.2,0.850}
\definecolor{lightBlue}{rgb}{0,0.4,0.7}
\definecolor{darkGreen}{rgb}{0.0,0.6,0.0}

% \usetheme{Antibes}      % schlichte gerade ecken, sehr dezent
% \usetheme{Berlin}       % kein groeßerer unterschied zu Antibes
% \usetheme{boxes}        % kein groeßerer unterschied zu Antibes
% \usetheme{default}      % kein groeßerer unterschied zu Antibes
% \usetheme{Dresden}      % kein groeßerer unterschied zu Antibes
% \usetheme{Luebeck}      % kein groeßerer unterschied zu Antibes
% \usetheme{Malmoe}       % kein groeßerer unterschied zu Antibes
% \usetheme{Montpellier}  % kein groeßerer unterschied zu Antibes
% \usetheme{Pittsburgh}   % kein groeßerer unterschied zu Antibes
% \usetheme{Rochester}    % kein groeßerer unterschied zu Antibes
% \usetheme{Szeged}       % kein groeßerer unterschied zu Antibes
% \usetheme{Bergen}       % Tehmen leiste links, als großer balken
% \usetheme{Berkeley}     % Navigations leiste am linken rand, sauber und uebersichtlich
% \usetheme{Boadilla}     % klares aussehen, ohne leisten oben
% \usetheme{CambridgeUS}  % 3D ausehen
% \usetheme{Copenhagen}   % 3D ausehen ohne schatten
% \usetheme{Darmstadt}    % 3D ausehen
% \usetheme{Frankfurt}    % 3D ausehen
% \usetheme{AnnArbor}     % 3D ausehen
% \usetheme{Ilmenau}      % 3D ausehen ohne schatten runde ecken
% \usetheme{JuanLesPins}  % 3D ausehen
% \usetheme{Madrid}       % 3D ausehen, ohne leiste oben
% \usetheme{PaloAlto}     % 3D ausehen Navigationsleiste links
% \usetheme{Warsaw}       % 3D ausehen
% \usetheme{Goettingen}   % Schlicht Navigationsleiste rechts.
% \usetheme{Hannover}     % Navigations leiste am linken rand, sauber und uebersichtlich
% \usetheme{Marburg}      % Schlicht Navigationsleiste rechts.
% \usetheme{Singapore}    % schlicht, Titel mittig zentriert, aufgeraeumt

%\usetheme[headheight=1cm,footheight=3cm]{boxes}
\usetheme{Frankfurt}


% Hier wird eine mögliche Variante verwendet.

% outer-color-themes: Namen von Seetieren: whale, seahorse, dolphin
\usecolortheme{dolphin}    % .., weiß, grau
% \usecolortheme{albatross}  % .., blau, dunkelblau
% \usecolortheme{beaver}     % weißer hintergrund, roter titel, grau unten
% \usecolortheme{beetle}     % .., grau, blau
% \usecolortheme{crane}      % weißer hintergrund, gelbe kaesten
% \usecolortheme{default}    % .., weiß, grau
% \usecolortheme{dove}       % weiß
% \usecolortheme{fly}        % grau
% \usecolortheme{lily}       % .., weiß, grau
% \usecolortheme{orchid}     % .., weiß, grau
% \usecolortheme{rose}       % .., weiß, grau
% \usecolortheme{seagull}    % .., weiß, grau
% \usecolortheme{seahorse}   % .., weiß, grau
% \usecolortheme{sidebartab} % .., weiß, grau
% \usecolortheme{structure}  % .., weiß, grau
% \usecolortheme{whale}      % .., weiß, grau
%\usecolortheme{wolverine}   % weiß, gelb, blau

% inner-color-themes: Namen von Blumen: lily, orchid, rose
\usecolortheme{dove}

% Theme für Schriften - es gibt noch weitere
%\usefonttheme{professionalfonts}
\usefonttheme{professionalfonts}

% inner-theme
%\useinnertheme{default}
\useinnertheme{rectangles}


% outer-theme
%\useoutertheme{infolines}
%\useoutertheme{smoothbars}
%\useoutertheme{miniframes}

\setbeamertemplate{footline}
{%
  \begin{beamercolorbox}[wd=1\textwidth,ht=3ex,dp=1.5ex,leftskip=.5em,rightskip=.5em]{author in head/foot}%
    \usebeamerfont{author in head/foot}%
\hfill\insertframenumber%
  \end{beamercolorbox}%
%\vspace*{-4.5ex}\hspace*{0.5\textwidth}%
%\begin{beamercolorbox}[wd=0.5\textwidth,ht=3ex,dp=1.5ex,left,leftskip=.5e m]{title in head/foot}%
%\usebeamerfont{title in head/foot}%
%\insertshorttitle%
%\end{beamercolorbox}%
}


\newcommand{\Name}[1]{\textit{\glqq#1\grqq}}

% \setbeamertemplate{frametitle}
% {
%   \vskip-0.25\beamer@headheight
%   \vskip-\baselineskip
%   \vskip-0.2cm
%   \hskip0.7cm\usebeamerfont*{frametitle}\insertframetitle
%   \vskip-0.10em
%   \hskip0.7cm\usebeamerfont*{framesubtitle}\insertframesubtitle
% }


%\usepackage{pgfpages}
%\setbeameroption{show notes on second screen}

\title{Preferences in answer set programming}
\author{Andreas Haselhuhn, Kai Trott}
\institute[UNI]{Universität Leipzig}
%\date{\today}


\begin{document}
\frame{\titlepage}
\section{Einleitung}

\begin{frame}
  Betrachtung von 3 Herangehensweisen f"ur ``Logik Programme mit Präferenzen''
  \begin{itemize}
    \item preferred alternating fixpoints
    \item compiling order preservation
    \item translations into standard logic programs
  \end{itemize}
\end{frame}


\begin{frame}{Logisches Programm}
  Ein ``general logic program'' ist eine Endliche Ansamlung von Regeln in der Form:
  \begin{figure}
    \begin{math}
      A_0~\leftarrow~A_1,~...,~A_m,~not~A_{m+1},~...,~not~A_n
    \end{math}
  \end{figure}
  bei dem gilt $n~\geq~m~\geq~0$\\[0.5cm]
  und jedes $A_i$ ist ein Atom.
\end{frame}


\begin{frame}{Erweitertes Logisches Programm}
  Ein ``extended logic program'' ist eine Endliche Ansamlung von Regeln in der Form:
  \begin{figure}
    \begin{math}
      L_0~\leftarrow~L_1,~...,~L_m,~not~L_{m+1},~...,~not~L_n
    \end{math}
  \end{figure}
  bei dem gilt $n~\geq~m~\geq~0$\\[0.5cm]
  und jedes $L_i$ ist ein Literal ($A$ oder $\neg A$).
\end{frame}


\begin{frame}{Negationen}
  schwache Negation:
  \begin{itemize}
    \item $cross~\leftarrow~not~train$
  \end{itemize}
  starke Negation:
  \begin{itemize}
    \item $cross~\leftarrow~\neg~train$
  \end{itemize}
\end{frame}


\begin{frame}{Definitionen}
  \begin{block}{\textbf{Define:} Lit}
    Lit ist die Menge aller Literale des Programms $\Pi$.
  \end{block}
  \begin{block}{\textbf{Define:}~Regel}
    $r = L_0~\leftarrow~L_1,~...,~L_m,~not~L_{m+1},~...,~not~L_n$
  \end{block}
  \begin{block}{\textbf{Define:}~$head$}
    $head(r) = L_0$
  \end{block}
  \begin{block}{\textbf{Define:}~$body$}
    $body(r) = {L_1,~...,~L_m,~not~L_{m+1},~...,~not~L_n}$
  \end{block}
\end{frame}


\begin{frame}{Definitionen}
  \begin{block}{\textbf{Define:}~$body^+$}
    $body^+(r) = {L_1,~...,~L_m}$
  \end{block}
  \begin{block}{\textbf{Define:}~$body^-$}
    $body^-(r) = {not~L_{m+1},~...,~not~L_n}$
  \end{block}
  \begin{block}{\textbf{Define:} $basic$}
    Ein Programm wird $basic$ genannt, wenn f"ur alle Regeln gilt:\\
    $body^-(r)~=~\emptyset$
  \end{block}
  \begin{block}{\textbf{Define:} $ground$}
    Eine Regel $r$ wird $ground$ genannt, wenn keine Variablen darin enthalten sind\\
  \end{block}
\end{frame}


\begin{frame}{Definitionen}
  \begin{block}{\textbf{Define:} $reduct$ von r}
    $r^+~=~head(r)~\rightarrow~body^+(r)$
  \end{block}
  \begin{block}{\textbf{Define:} $reduct$ von $\Pi$}
    $\Pi$ in relation zu $X$. $X$ ist eine Menge von Literalen\\
    $\Pi^X~=\{r^+~|~r \in \Pi~and~body^-(r) \cap X = \emptyset \}$
  \end{block}
\end{frame}


\begin{frame}{Definitionen}
  \begin{block}{\textbf{Define:} $logically~closed$}
    We say that $X$ is logically closed iff it is either consistent or equals Lit.
  \end{block}
  \begin{block}{\textbf{Define:} $closed~under~a~basic~program$}
    A set of literals $X$ is closed under a basic program $\Pi$ iff for
    any $r~\in~\Pi$, $head(r)~\in~X$ whenever $body^+(r) \subseteq X$.
  \end{block}
\end{frame}


\begin{frame}{Definitionen}
  \begin{block}{\textbf{Define:} Operator $Cn(\Pi)$}
  $Cn(\Pi)$ bezeichnet die kleinste Menge von Literalen die sowohl
  $logically~closed$, als auch $closed~under~a~basic~program~\Pi$ ist
  \end{block}
  \begin{block}{\textbf{Define:} $answer~set$}
    $Cn(\Pi^X)=X$
  \end{block}
\end{frame}



\section{fixpoint}

\begin{frame}
  \begin{block}{\textbf{Definition 1:}}
    Let $(\Pi, <)$ be an ordered
    logic program and let $X$ be a set of literals. We define
    \begin{itemize}
      \item $X_0 = \emptyset$~~~and for $i \geq 0$
      \item $X_{i+1} = X_i \cup \{head(r)~|~$
          \begin{itemize}
            \item[I.] $r \in \Pi$ is active wrt $(X_i, X)$ and
            \item[II.] there is no rule $r' \in \Pi$ with $r < r'$ such that
              \begin{itemize}
                \item[(a)] $r'$ is active wrt $(X, X_i) and$
                \item[(b)] $head(r') \notin X_i$
              \end{itemize}
          \end{itemize}
      \end{itemize}
    \end{block}
  \end{frame}

\section{Compiling order preservation}

\begin{frame}
  \begin{itemize}
    \item "Ubersetzung von geordneten logischen Programmen $\Pi$ zu standard logischen Programmen $\Pi'$
    \item akzeptiert nur Antworten von $\Pi$ die $order~preserving$ sind
  \end{itemize}
\end{frame}


\begin{frame}
  \begin{block}{\textbf{Define:} $order~preserving$}
    Let $(\Pi, <)$ be a statically ordered program and
    let $X$ be an answer set of $\Pi$.
    Then, $X$ is called <-preserving, if there exists an enumer-
    ation $(r_i)_{i \in I}~of~\Gamma^{X}_{\Pi}$  such that for every $i, j \in I$ we have that:
    \begin{enumerate}
      \item $body^+(r') \subseteq \{head(r_j)~|~j < i\};~and$
      \item $if~r_i < r_j,~then~j < i;~and$
      \item $if~r_i < r',~and~r' \in \Pi\setminus\Gamma^{X}_{\Pi},~then$
        \begin{itemize}
          \item[(a)] $body^+(r') \nsubseteq X~or$
          \item[(b)] $body^-(r') \cap \{head(r_j)~|~j < i\} \neq \emptyset$
        \end{itemize}
    \end{enumerate}
  \end{block}
\end{frame}

\include{preference}
\include{relationships}
\end{document}

