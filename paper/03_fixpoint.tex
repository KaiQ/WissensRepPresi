\section{Alternating Fixpoints}

Die Notation von \emph{Answer Sets} (ohne Präferenzen) basiert auf
der Reduktion von erweiterten logischen Programmen auf \emph{basic} Programme.
Dadurch ist es jedoch nicht mehr möglich einen Konflikt zu berücksichtigen,
da alle Konflikte simultan gelöst werden beim überführen von $\Pi$ zu $\Pi^X$.
Das lösen von Konflikten muss daher anhand der ursprünglichen Regeln Geschehen
um Blockaden innerhalb der Reglen berücksichtigen zu können.
Ohne den Nagativen Teil $body^-(r)$ kann nicht mehr nachgewiesen werden, ob
$head(r') \in body^-(r)$ im Fall von $r < r'$ hält.\\
Ein Ansatz hierfür wurde von \cite{wang2000alternating} verfolgt.
Dieser beruht auf dem Konzept von \emph{active}.\\

\begin{definition}
  Let $(\Pi, <)$ be an ordered logic program and let $X$ be a set of literals.
  We define:\\
  \begin{equation*}
    \begin{split}
      X_0 & = \emptyset\hspace*{0.3cm} \text{~and for~} i \geq 0 \\
      X_{i+1} & = X_i \cup \{ head(r)~| \\
        & \hspace*{0.4cm}\left. \begin{aligned}
          I. &\hspace*{0.2cm} r \in \Pi \text{~is active wrt~} (X_i, X) \text{~and} \\
          II. &\hspace*{0.2cm} \text{there is no rule~} r' \in \Pi \text{~with~} r < r'
          \text{~such that} \\
          & \hspace*{0.2cm} \begin{aligned}
            (a) &\hspace*{0.2cm} r' \text{~is active wrt~} (X, X_i) \text{~and} \\
            (b) &\hspace*{0.2cm} head(r') \notin X_i
          \end{aligned}
        \end{aligned}
      \right \}
    \end{split}
  \end{equation*}
  Then, $C_{(\Pi, <)}(X) = \bigcup_{i\geq 0} X_i$ if $\bigcup_{i\geq 0} X_i$ is
  consistent. \\Otherwise, $C_{(\Pi, <)}(X) = Lit$.
  \label{def:1}
\end{definition}

Die grundlegende Idee daran ist es eine Regel $r$ nur dann anzuwenden, wenn alle
existierenden höherwärtigen Regeln $r'$ schon auf Anwendbarkeit geprüft wurden.
Dies ist dann zutreffend wenn die Vorraussetzungen derer nicht erfüllbar sind, also
$body(r') \not\subseteq X$, oder $r'$ wird durch ein schon abgeleitetes blockiert
$body^-(r) \cap X_i \neq \emptyset$, oder $r'$ beziehungsweise eine andere Regel mit
dem selben Kopf wurde bereits angewandt $head(r') \in X_i$.\\
Genau so wie $C_\Pi$ ist auch der Operator $C_{(\Pi, <)}$ nicht Monoton.
Folglich wird für jedes Set $X \subseteq Lit$ die \emph{alternating transformation}
der Form $(\Pi, <)$ als $A_{(\Pi, <)}(X) = C_{(\Pi, <)}(C_{(\Pi, <)}(X))$ definiert.
Ein Fixpunkt von $A_{(\Pi, <)}$ wird als \emph{alternating fixpoint} von $(\Pi, <)$
bezeichnet. Hierbei sei zu beachten, das $A_{(\Pi, <)}$ monoton ist.\\



Zur verdeutlichung steht das folgende Beispiel:\\
  \begin{example}[$\Pi_2, <$]
%\begin{minipage}{0.8\textwidth}
    \begin{align*}
      r_1: && \neg f \hspace*{0.2cm}&\leftarrow \hspace*{0.2cm}p, not~f &  r_2 < r_1\\
      r_2: && w \hspace*{0.2cm}&\leftarrow \hspace*{0.2cm}b, not~\neg w \\
      r_3: && f \hspace*{0.2cm}&\leftarrow \hspace*{0.2cm}w, not~\neg f \\
      r_4: && b \hspace*{0.2cm}&\leftarrow \hspace*{0.2cm}p \\
      r_5: && p \hspace*{0.2cm}&\leftarrow
    \end{align*}
    \label{example:pi2}
%\end{minipage}
  \end{example}



\begin{align*}
  X_0 &= \emptyset           & X_0' &= \emptyset \\
  X_1 &= \{p\}               & X_1' &= \{p\} \\
  X_2 &= \{p, b, \neg f\}    & X_2' &= \{p, b\} \\
  X_3 &= \{p, b, \neg f, w\} & X_3' &= X_2' \not = X' \\
  X_4 &= X_3 = X
\end{align*}








