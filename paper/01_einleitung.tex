\section{Einleitung}
Diese Ausarbeitung bezieht sich auf das Paper \glqq A Comparative Study of Logic Programs with Preference\grqq\cite{SchaubWang}.
Sie stellt eine Übersetzung so wie eine etwas weitere Einführung in das Thema dar.
Es wird ein Überblick über Logikprogramme mit Präferenzen geliefert und eine kurze Einführung in die benötigten Definitionen und Notationen gegeben.
Sowie ein Einblick in die verwendeten Sprachkonstrukte und deren Eigenschaften dargestellt.
Des weiteren möchten wir drei verschiedene Herangehensweisen für \glqq Logik Programme mit Präferenzen\grqq~ vorstellen:
\begin{itemize}
  \item preferred alternating fixpoints (W-preference)
  \item compiling order preservation (D-preference)
  \item translations into standard logic programs (B-preference)
\end{itemize}
Diese diese werden im weiteren mit einander verglichen beziehungsweise es werden Unterschiede aufgezeigt.
