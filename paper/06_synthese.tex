\section{Synthesis}

In den Kapiteln \ref{sec:fixpoint} und \ref{sec:order} sind drei verschiedene
Verfahren zur Charakterisierung von \emph{answer sets} gezeigt wurden.
Trotz der unterschiedlichen Herangehensweisen wurden ähnliche
\emph{answer sets} ausgewählt.

\subsection{D-preference}
Für die \emph{D-preference} wird eine \emph{fixpoint} Definition gebildet.
Hierfür wird von einer bijunktiven Abbildungsregel $rule(\cdot)$ von
Regelköpfen zu Regeln $rule(headr)) = r$ ausgegangen, entsprechend
$rule({head(r)~|~r \in R}) = R$. Solche Abbildungen können in einer
disjunktiven Form durch charakterisierung verschiedener Aufkommen
von Literalen definiert werden.

\begin{definition}
  Let $(\Pi, <)$ be an ordered logic program and let $X$ be a set of literals.
  We define:\\
  \begin{equation*}
    \begin{split}
      X_0 & = \emptyset\hspace*{0.3cm} \text{~and for~} i \geq 0 \\
      X_{i+1} & = X_i \cup \{ head(r)~| \\
        & \hspace*{0.4cm}\left. \begin{aligned}
          I. &\hspace*{0.2cm} r \in \Pi \text{~is active wrt~} (X_i, X) \text{~and} \\
          II. &\hspace*{0.2cm} \text{there is no rule~} r' \in \Pi \text{~with~} r < r'
          \text{~such that} \\
          & \hspace*{0.2cm} \begin{aligned}
            (a) &\hspace*{0.2cm} r' \text{~is active wrt~} (X, X_i) \text{~and} \\
            (b) &\hspace*{0.2cm} r' \not\in rule(X_i)
          \end{aligned}
        \end{aligned}
      \right \}
    \end{split}
  \end{equation*}
  Then, $C^D_{(\Pi, <)}(X) = \bigcup_{i\geq 0} X_i$ if $\bigcup_{i\geq 0} X_i$ is
  consistent. \\Otherwise, $C^D_{(\Pi, <)}(X) = Lit$.
  \label{def:fix_d}
\end{definition}

Der Unterschied zwischen Definition \ref{def:fix_d} und Definition \ref{def:1}
liegt in Punnkt IIb. Die \emph{D-preference} verlangt das eine höherwärtige
Regel vorher effektiv angewendet wurden ist, wobei bei der \emph{W-preference}
nur der Kopf der Regel präsent sein muss, ungeachtet dessen von welcher Regel
dieser geliefert wurde.\\
Dies demonstriert das Programm $(\Pi_{\ref{example:pi3}}, <)$

\begin{example}[$\Pi_{\ref{example:pi3}}, <$]
  \begin{align*}
    r_1: && a \hspace*{0.2cm}&\leftarrow \hspace*{0.2cm} not~b &  r_2 < r_1\\
    r_2: && b \hspace*{0.2cm}&\leftarrow\\
    r_3: && a \hspace*{0.2cm}&\leftarrow
  \end{align*}
  \label{example:pi3}
\end{example}

Es existiert nur ein \emph{W-preferred} \emph{answer set} $\{a, b\}$ und kein
\emph{D-preferred}. Das selbte trifft zu wenn bei Programm \ref{example:pi3}
die Regel $r$ durch $r':~a\leftarrow b$ ersetzt wird.


\subsection{W-preference}
Eine \emph{W-preference} kann im Ausdruck einer \emph{order preserving}
Charakterisiert werden.

\begin{definition}
  Let $(\Pi, <)$ be a statically ordered program and let X be an answer set of $\Pi$.
  Then, X is called $<^W$-preserving, if there exists an enumeration
  $\langle r_i \rangle_{i \in I}$~of~$\Gamma_{\Pi}^X$ such that for every $i, j \in I$
  we have that:
  \begin{itemize}
    \item[0.] 
      \begin{itemize}
        \item[(a)] $body^+(r_i) \subseteq \{ head(r_j)~|~j<i\}$ or
        \item[(b)] $head(r_i) \in \{ head(r_j)~|~j<i\}$; and
      \end{itemize}
    \item[1.] if $r_i < r_j$, then $j<i$; and \\
    \item[2.] if $r_i < r'$ and $r' \in \Pi \backslash \Gamma_{\Pi}^X$, then
      \begin{itemize}
        \item[(a)] $body^+(r') \not \subseteq X$ or
        \item[(b)] $body^-(r') \cap \{head(r_j)~|~j<i\} \not = \emptyset$ or
        \item[(c)] $head(r') \in \{head(r_j)~|~j<i\}$
      \end{itemize}
  \end{itemize}
\end{definition}

Der hauptsächliche Unterschied deses Konzeptes der \emph{order preserving} zum
originalen besteht in der abgeschwächten Darstellung der \emph{groundedness}.
Die beinhaltet die Regeln in $\Gamma^{X}_{\Pi}$ (über Bedingung 0b) als auch jene
aus $\Pi \backslash \Gamma^{X}_{\Pi}$. Der Rest der Definition unterscheidet sich
nicht von Definition \ref{def:2}.\\
Beispielsweise wird das \emph{answer set} $\{a, b\}$ aus Beispiel
\ref{example:pi3} durch die $<^W$-preserving Regelsequenz
$\langle r_3, r_2 \rangle$ erzeugt. hierbei ist zu beachten, dass $r_1$ die
Bedingung 2c erfüllt, jedoch nicht 2a noch 2b.\\
Diese abgeschwächten Darstellung der \emph{groundedness} kann einfach in die
Umwandlung von Definition \ref{def:3} integriert werden.

\begin{definition}
  Given the same prerequisites as in definition \ref{def:3}.
  Then, the logic program $\mathcal{T}^W(\Pi)$ over $L^+$ is defined as\\
  $\mathcal{T}^W(\Pi) = \bigcup_{r \in \Pi}\tau(r) \cup \{c_5(r, r')~|~r, r' \in \Pi\}$
  , where
    \begin{align*}
      c_5(r, r'): && ok'(n_r, n_{r'}) \hspace*{0.2cm}&\leftarrow \hspace*{0.2cm} (n_r \prec n_{r'}), head(r')
    \end{align*}
\end{definition}

Der Sinn von $c_5(r, r')$ ist es Regeln des Präferenz Prozesses auszuschließen,
nachdem der Kopf hinzugefügt wurde.
