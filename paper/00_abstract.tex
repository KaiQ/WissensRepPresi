\begin{abstract}
  Zu Beginn werden drei Verfahren für das Handhaben von
  Präferenzen in logischen Programmen im Framework des
  \emph{answer set programming} untersucht.
  Da diese Verfahren unterschiedliche Formale Mittel
  verwenden, werden einheitliche Charakteristiken verwendet
  um eine einsicht über die Verhältnisse untereinander zu erhalten.
  Präziser werden die verschiedenen Charakteristiken in der Form von
  \begin{inparaenum}[(i)]
    \item \emph{fixpoints},
    \item \emph{order preserving}, und
    \item \emph{translations into standard logic programs}
  \end{inparaenum} gezeigt.
  Während die beiden Umformungen Semantiken für logische Programme
  mit Präferenzinformationen beinhalten, implementiert die letzte
  Techniken für diese Annäherungen.
\end{abstract}
