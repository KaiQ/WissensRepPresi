% LLNCStmpl.tex
% Template file to use for LLNCS papers prepared in LaTeX 2e
%http://www.springer.com/lncs

\documentclass{llncs}
\usepackage{etex}
\usepackage{amsmath}
%Use this line instead if you want to use running heads (i.e. headers on each page):
%\documentclass[runningheads]{llncs}

\usepackage[utf8]{inputenc}
\usepackage[ngerman]{babel}
\usepackage{bibgerm} % deutsches Literaturvereichnis
\usepackage[colorlinks=false, pdfborder={0 0 0}]{hyperref}
\usepackage{color}
\usepackage{graphicx}
\usepackage{tikz}
\usepackage{ctable}
\usepackage{multirow}
\usetikzlibrary{arrows,shapes,positioning,shadows,trees,decorations.pathreplacing}
\definecolor{gray}{gray}{0.45}
\newcommand{\zeit}[1]{{\color{gray} (#1)}}



\begin{document}
\title{Preferences in answer set programming}

%If you're using runningheads you can add an abreviated title for the running head on odd pages using the following
%\titlerunning{abreviated title goes here}
%and an alternative title for the table of contents:
%\toctitle{table of contents title}

%\subtitle{Subtitle Goes Here}

%For a single author
\author{Andreas Haselhuhn, Kai Trott}
  
%For multiple authors:
%\author{First Author Name\inst{1} \and Second Author Name\inst{2}}


%If using runnningheads you can abbreviate the author name on even pages:
%\authorrunning{abbreviated author name}
%and you can change the author name in the table of contents
%\tocauthor{enhanced author name}

%For a single institute
%\institute{Universität Leipzig \email{trott@studserv.uni-leipzig.de}}
\institute{Universität Leipzig}
%\institute{}

% If authors are from different institutes 
%\institute{First Institute Name \email{email address} \and Second Institute Name\thanks{Thank you to...} \email{email address}}

%to remove your email just remove '\email{email address}'
% you can also remove the thanks footnote by removing '\thanks{Thank you to...}'


\maketitle

\section{Einleitung}
Diese Ausarbeitung bezieht sich auf das Paper von \cite{SchaubWang}. Sie stellt eine Übersetzung so wie eine etwas weitere Einführung in das Thema dar. Es wird ein Überblick über Logikprogramme mit Präferenzen geliefert und eine kurze Einführung in die benötigten Definitionen und Notationen gegeben. Sowie ein Einblick in die verwendeten Sprachkonstrukte und deren Eigenschaften dargestellt. Des weiteren möchten wir drei verschiedene Herangehensweisen für ``Logik Programme mit Präferenzen'' vorstellen:
\begin{itemize}
  \item preferred alternating fixpoints (W-preference)
  \item compiling order preservation (D-preference)
  \item translations into standard logic programs (B-preference)
\end{itemize}
Diese diese werden im weiteren mit einander verglichen beziehungsweise es werden Unterschiede aufgezeigt.
\section{Definitionen}

Betrachtung von 3 Herangehensweisen f"ur ``Logik Programme mit Präferenzen''
\begin{itemize}
  \item preferred alternating fixpoints
  \item compiling order preservation
  \item translations into standard logic programs
\end{itemize}


Ein ``general logic program'' ist eine Endliche Ansamlung von Regeln in der Form:
\begin{figure}
  \begin{math}
    A_0~\leftarrow~A_1,~...,~A_m,~not~A_{m+1},~...,~not~A_n
  \end{math}
\end{figure}
bei dem gilt $n~\geq~m~\geq~0$\\[0.5cm]
und jedes $A_i$ ist ein Atom.


Ein ``extended logic program'' ist eine Endliche Ansamlung von Regeln in der Form:
\begin{figure}
  \begin{math}
    L_0~\leftarrow~L_1,~...,~L_m,~not~L_{m+1},~...,~not~L_n
  \end{math}
\end{figure}
bei dem gilt $n~\geq~m~\geq~0$\\[0.5cm]
und jedes $L_i$ ist ein Literal ($A$ oder $\neg A$).


ein todo ist ein paar aus extended log prog und relation die element von logik programm....\\
$<~\subseteq~\Pi~x~\Pi$\\
$r1, r2 \in \Pi$
keine selbstabbildung $r1 \neq r1$


schwache Negation:
\begin{itemize}
  \item $cross~\leftarrow~not~train$
\end{itemize}
starke Negation:
\begin{itemize}
  \item $cross~\leftarrow~\neg~train$
\end{itemize}


{\textbf{Define:} Lit}
  Lit ist die Menge aller Literale des Programms $\Pi$.

{\textbf{Define:}~Regel}
  $r = L_0~\leftarrow~L_1,~...,~L_m,~not~L_{m+1},~...,~not~L_n$

{\textbf{Define:}~$head$}
  $head(r) = L_0$

{\textbf{Define:}~$body$}
  $body(r) = {L_1,~...,~L_m,~not~L_{m+1},~...,~not~L_n}$



{\textbf{Define:}~$body^+$}
  $body^+(r) = {L_1,~...,~L_m}$

{\textbf{Define:}~$body^-$}
  $body^-(r) = {not~L_{m+1},~...,~not~L_n}$

{\textbf{Define:} $basic$}
  Ein Programm wird $basic$ genannt, wenn f"ur alle Regeln gilt:\\
  $body^-(r)~=~\emptyset$



{\textbf{Define:} $active$ r}
  $body^+(r) \subseteq X$ and $body^-(r) \cap Y = \emptyset$

{\textbf{Define:} $reduct$ von r}
  $r^+~=~head(r)~\leftarrow~body^+(r)$

{\textbf{Define:} $reduct$ von $\Pi$}
  $\Pi$ in relation zu $X$. $X$ ist eine Menge von Literalen\\
  $\Pi^X~=\{r^+~|~r \in \Pi~and~body^-(r) \cap X = \emptyset \}$



{\textbf{Define:} $logically~closed$}
  We say that $X$ is logically closed iff it is either consistent or equals Lit.

{\textbf{Define:} $closed~under~a~basic~program$}
  A set of literals $X$ is closed under a basic program $\Pi$ iff for
  any $r~\in~\Pi$, $head(r)~\in~X$ whenever $body^+(r) \subseteq X$.



{\textbf{Define:} $Cn(\Pi)$}
  $Cn(\Pi)$ bezeichnet die kleinste Menge von Literalen die sowohl
  $logically~closed$, als auch $closed~under~a~basic~program~\Pi$ ist

{\textbf{Define:} $answer~set$}
  $Cn(\Pi^X)=X$



\section{Alternating Fixpoints}

Die Notation von Answer Sets (ohne Präferenzen) basiert auf
der Reduktion von erweiterten logischen programmen auf basic Programme.
Dadurch ist es jedoch nicht mehr möglich einen Konflikt zu berücksichtigen,
da alle Konflikte simultan gelöst werden beim überführen von $\Pi$ zu $\Pi^X$.
Das lösen von Konflikten muss daher anhand der ursprünglichen Regeln Geschehen
um Blockaden innerhalb der Reglen berücksichtigen zu können.
Ohne den Nagativen Teil $body^-(r)$ kann nicht mehr nachgewiesen werden, ob
$head(r') \in body^-(r)$ im Fall von $r < r'$ hält.

\begin{definition}
  Let $(\Pi, <)$ be an ordered logic program and let $X$ be a set of literals.
  We define:\\
  \begin{equation*}
    \begin{split}
      X_0 & = \emptyset\hspace*{0.3cm} \text{~and for~} i \geq 0 \\
      X_{i+1} & = X_i \cup \{ head(r)~| \\
        & \hspace*{0.4cm}\left. \begin{aligned}
          I. &\hspace*{0.2cm} r \in \Pi \text{~is active wrt~} (X_i, X) \text{~and} \\
          II. &\hspace*{0.2cm} \text{there is no rule~} r' \in \Pi \text{~with~} r < r'
          \text{~such that} \\
          & \hspace*{0.2cm} \begin{aligned}
            (a) &\hspace*{0.2cm} r' \text{~is active wrt~} (X, X_i) \text{~and} \\
            (b) &\hspace*{0.2cm} head(r') \notin X_i
          \end{aligned}
        \end{aligned}
      \right \}
    \end{split}
  \end{equation*}
  Then, $C_{(\Pi, <)}(X) = \bigcup_{i\geq 0} X_i$ if $\bigcup_{i\geq 0} X_i$ is
  consistent. \\Otherwise, $C_{(\Pi, <)}(X) = Lit$.
\end{definition}

Text nachher Lorem ipsum...

\section{Compiling order preservation}

Eine Übersetzung eines geordneten logischen Programmes $(\Pi, <)$ in ein
Standard logisches Programm $\Pi'$ wurde in \cite{delgrande2000logic} entwickelt.
Dieser Vorgang stellt sicher, das die aus $\Pi'$ resultierenden \emph{answer sets}
allen \emph{order preserving answer sets} von $\Pi$ entsprechen.

\begin{definition}
  Let $(\Pi, <)$ be a statically ordered program and let X be an answer set of $\Pi$.
  Then, X is called <-preserving, if there exists an enumeration
  $\langle r_i \rangle_{i \in I}$~of~$\Gamma_{\Pi}^X$ such that for every $i, j \in I$
  we have that:
  \begin{itemize}
    \item[0] $body^+(r_i) \subseteq \{ head(r_j)~|~j<i\}$ and \\
    \item[1] if $r_i < r_j$, then $j<i$ and \\
    \item[2] if $r_i < r'$ and $r' \in \Pi \backslash \Gamma_{\Pi}^X$, then
      \begin{itemize}
        \item[(a)] $body^+(r') \not \subseteq X$ or
        \item[(b)] $body^-(r') \cap \{head(r_j)~|~j<i\} \not = \emptyset$
      \end{itemize}
  \end{itemize}
\end{definition}

Die Bedingung 0 macht die Eigenschaft der \emph{groundedness} explizit.\\
Bedingung 1 garantiert die Kompatibilität von $\langle r_i \rangle_{i \in I}$
zu $<$.\\
Bedingung 2 ist vergleichbar mit \emph{Condition II} aus Definition \ref{def:1}.
Durch sie können höherwertige Reglen nie durch niederwertige Blockiert werden.\\
Entsprechend \emph{Example} \ref{example:pi2} ist $X = \{p, b, \not f, w\}$ das
einzige \emph{<-preserving} \emph{answer set} von $\Pi_2$. Es kann durch die 
\emph{grounded} Sequenzen
$\langle r_5, r_4, r_1, r_2 \rangle$ und $\langle r_5, r_1, r_4, r_2 \rangle$
erzeugt werden, welche beide die Bedingungen 1 und 2 erfüllen.
Die einzige \emph{grounded} Sequenz die $X' = \{p, b, f, w\}$ generiert,
$\langle r_5, r_4, r_2, r_3 \rangle$, verletzt Bedingung 2b.
Die entsprechende Übersetzung verbindet die Information der Ordnung
mit dem logischen Programm über das spezielle Prädikaten Symbol $\prec$.
Dies erlaubt ebenfalls eine Ordnung in einer Dynamischen Form.\\
Ein logisches Programm über einer propositionalen Sprache $L$ ist genau
dann dynamisch geordnet genannt, wenn $L$ die folgenden paarweisen disjunkten
Kategorien beinhaltet:
\begin{itemize}
  \item[(i)]   eine Menge $N$, welche die Namen der Regeln beinhaltet
  \item[(ii)]  eine Menge $At$ von Atomen eines Programmes
  \item[(iii)] eine Menge $At_{\prec}$ von präferenz Atomen $s \prec t$,
    wobei $s, t \in N$ Namen sind.
\end{itemize}
Für jedes Programm $\Pi$ existiert eine bijective Funktion $n(\cdot)$ welche jeder
Regel $r \in \Pi$ einen Namen $n(r) \in N$ zuweist.
Für eine vereinfachte Schreibweise wird $n_r$ anstatt $n(r)$ verwendet.
Gegebenen falls auch $n_i$ statt $n_{r_i}$.\\
Ein Atom $n_r \prec n_{r'} \in At_{\prec}$  lässt davon ausgehen, dass
$r < r'$ gilt.
Ein statisch geordnetes Programm $(\Pi, <)$ kann dem entsprechend durch ein
Programm, welches präferenz Atome nur anhand derer Fakten beinhaltet,
beschrieben werden. Die Schreibweise ist dann
$\Pi \cup \{(n_r \prec n_{r'}) \leftarrow~|~r < r'\}$.\\
Bei gegebenem $r < r'$ muss sichergestellt sein, das $r'$ vor $r$ berücksichtigt
wird. In dem Sinne, das bei einem \emph{answer set} X vor dem anwenden der
Regel $r$, zuerst die Regel $r'$ angewandt oder wiederlegt worden ist.
Dies wird durch das umformen der Regeln erziehlt, so das die Reihenfolge
der Anwendung der Regeln explizit kontrolliert werden kann.
Dadurch verfällt die Notwendigkeit vor dem anwenden von Regeln zu prüfen,
ob eine andere Regel vorher angewendet werden muss oder ungültig wird.
Für eine Regel $r$ existieren zwei Möglichkeiten das sie nich angewendet werden
müssen:
\begin{itemize}
  \item ein Literal aus $head^+(r)$ kommt nicht im \emph{answer set} vor, oder
  \item ein Literal aus $head^-(r)$ ist im \emph{answer set}
\end{itemize}
Zum ausdrücken einer Blockierung für jeder Regel $r$ in dem gegebenen Programm $\Pi$
wird ein neues Atom $bl(n_r)$ eingeführt.
Vergleichsweise dazu wird das Atom $ap(n_r)$ eingeführt, um eine mögliche anwendung
der Regel auszudrücken.
Zur Kontrolle der Reihenfolge der Anwendung wird das Atom $ok(n_r)$ eingeführt.
Unformell heißt dies das es dann \emph{ok} ist eine Regel $r$ anzuwenden, wenn
es \emph{ok} ist in bezug auf jede höherwertige Regel $r'$.
Dies ist dann der Fall, wenn die höherwertige Regel $r'$ blockiert oder angewandt
wurde.\\
Formal betrachtet, bei einem gegebenen dynamisch geordneten Programm $\Pi$ über
$L$, dann sei $L^+$ eine Sprache erreicht durch hinzufügen von, für alle
$r, r' \in \Pi$, neuen paarweisen disjunkten propositionalen Atomen
$ap(n_r)$, $bl(n_r)$, $ok(n_r)$ und $ok'(n_r, n_{r'})$.
Dann stellt die Übersetzung $\Gamma$ ein geordnetes Programm $\Pi$ über $L$ in
ein Standard Programm $\Gamma(\Pi)$ über $L^+$ wie folgt dar.

\begin{definition}
  Let $\Pi = \{r_1, ..., r_k\}$ be a dynamically ordered logic program over $L$.
  Then, the logic program $\Gamma(\Pi)$ over $L^+$ is defined as
  $\Gamma(\Pi) = \bigcup_{r \in \Pi^\Gamma}(r)$, where $\Gamma(r)$ consists of the
  following rules, for $L^+ \in body^+(r)$, $L^- \in body^-(r)$, and $r'$,
  $r'' \in \Pi:$\\
  \begin{minipage}{0.8\textwidth}
    \begin{align*}
      a_1(r): && head(r) \hspace*{0.2cm}&\leftarrow \hspace*{0.2cm} ap(n_r)\\
      a_2(r): && ap(n_r) \hspace*{0.2cm}&\leftarrow \hspace*{0.2cm} ok(n_r), body(r)\\
      b_1(r, L^+): && bl(n_r) \hspace*{0.2cm}&\leftarrow \hspace*{0.2cm} ok(n_r), not~L^+\\
      b_2(r, L^-): && bl(n_r) \hspace*{0.2cm}&\leftarrow \hspace*{0.2cm} ok(n_r), L^-\\
      \\
      c_1(r): && ok(n_r) \hspace*{0.2cm}&\leftarrow \hspace*{0.2cm} ok'(n_r, n_{r_1}), ..., ok'(n_r, n_{r_k})\\
      c_2(r, r'): && ok'(n_r, n_{r'}) \hspace*{0.2cm}&\leftarrow \hspace*{0.2cm} not~(n_r \prec n_{r'})\\
      c_3(r, r'): && ok'(n_r, n_{r'}) \hspace*{0.2cm}&\leftarrow \hspace*{0.2cm} (n_r \prec n_{r'}), ap(n_{r'})\\
      c_4(r, r'): && ok'(n_r, n_{r'}) \hspace*{0.2cm}&\leftarrow \hspace*{0.2cm} (n_r \prec n_{r'}), bl(n_{r'})\\
      \\
      t(r, r', r''): && n_r \prec n_{r''} \hspace*{0.2cm}&\leftarrow \hspace*{0.2cm} n_r \prec n_{r'}, n_{r'} \prec n_{r''}\\
      as(r, r'): && \neg (n_{r'} \prec n_{r}) \hspace*{0.2cm}&\leftarrow \hspace*{0.2cm} n_r \prec n_{r'}
    \end{align*}
  \end{minipage}
\end{definition}



\section{Brewka and Eiter's concept of preference}


\section{Synthesis}

In den Kapiteln \ref{sec:fixpoint} und \ref{sec:order} sind drei verschiedene
Verfahren zur Charakterisierung von \emph{answer sets} gezeigt wurden.
Trotz der unterschiedlichen Herangehensweisen wurden ähnliche
\emph{answer sets} ausgewählt.

\subsection{D-preference}
Für die \emph{D-preference} wird eine \emph{fixpoint} Definition gebildet.
Hierfür wird von einer bijunktiven Abbildungsregel $rule(\cdot)$ von
Regelköpfen zu Regeln $rule(headr)) = r$ ausgegangen, entsprechend
$rule({head(r)~|~r \in R}) = R$. Solche Abbildungen können in einer
disjunktiven Form durch charakterisierung verschiedener Aufkommen
von Literalen definiert werden.

\begin{definition}
  Let $(\Pi, <)$ be an ordered logic program and let $X$ be a set of literals.
  We define:\\
  \begin{equation*}
    \begin{split}
      X_0 & = \emptyset\hspace*{0.3cm} \text{~and for~} i \geq 0 \\
      X_{i+1} & = X_i \cup \{ head(r)~| \\
        & \hspace*{0.4cm}\left. \begin{aligned}
          I. &\hspace*{0.2cm} r \in \Pi \text{~is active wrt~} (X_i, X) \text{~and} \\
          II. &\hspace*{0.2cm} \text{there is no rule~} r' \in \Pi \text{~with~} r < r'
          \text{~such that} \\
          & \hspace*{0.2cm} \begin{aligned}
            (a) &\hspace*{0.2cm} r' \text{~is active wrt~} (X, X_i) \text{~and} \\
            (b) &\hspace*{0.2cm} r' \not\in rule(X_i)
          \end{aligned}
        \end{aligned}
      \right \}
    \end{split}
  \end{equation*}
  Then, $C^D_{(\Pi, <)}(X) = \bigcup_{i\geq 0} X_i$ if $\bigcup_{i\geq 0} X_i$ is
  consistent. \\Otherwise, $C^D_{(\Pi, <)}(X) = Lit$.
  \label{def:fix_d}
\end{definition}

Der Unterschied zwischen Definition \ref{def:fix_d} und Definition \ref{def:1}
liegt in Punnkt IIb. Die \emph{D-preference} verlangt das eine höherwärtige
Regel vorher effektiv angewendet wurden ist, wobei bei der \emph{W-preference}
nur der Kopf der Regel präsent sein muss, ungeachtet dessen von welcher Regel
dieser geliefert wurde.\\
Dies demonstriert das Programm $(\Pi_{\ref{example:pi3}}, <)$

\begin{example}[$\Pi_{\ref{example:pi3}}, <$]
  \begin{align*}
    r_1: && a \hspace*{0.2cm}&\leftarrow \hspace*{0.2cm} not~b &  r_2 < r_1\\
    r_2: && b \hspace*{0.2cm}&\leftarrow\\
    r_3: && a \hspace*{0.2cm}&\leftarrow
  \end{align*}
  \label{example:pi3}
\end{example}

Es existiert nur ein \emph{W-preferred} \emph{answer set} $\{a, b\}$ und kein
\emph{D-preferred}. Das selbte trifft zu wenn bei Programm \ref{example:pi3}
die Regel $r$ durch $r':~a\leftarrow b$ ersetzt wird.


\subsection{W-preference}
Eine \emph{W-preference} kann im Ausdruck einer \emph{order preserving}
Charakterisiert werden.

\begin{definition}
  Let $(\Pi, <)$ be a statically ordered program and let X be an answer set of $\Pi$.
  Then, X is called $<^W$-preserving, if there exists an enumeration
  $\langle r_i \rangle_{i \in I}$~of~$\Gamma_{\Pi}^X$ such that for every $i, j \in I$
  we have that:
  \begin{itemize}
    \item[0.] 
      \begin{itemize}
        \item[(a)] $body^+(r_i) \subseteq \{ head(r_j)~|~j<i\}$ or
        \item[(b)] $head(r_i) \in \{ head(r_j)~|~j<i\}$; and
      \end{itemize}
    \item[1.] if $r_i < r_j$, then $j<i$; and \\
    \item[2.] if $r_i < r'$ and $r' \in \Pi \backslash \Gamma_{\Pi}^X$, then
      \begin{itemize}
        \item[(a)] $body^+(r') \not \subseteq X$ or
        \item[(b)] $body^-(r') \cap \{head(r_j)~|~j<i\} \not = \emptyset$ or
        \item[(c)] $head(r') \in \{head(r_j)~|~j<i\}$
      \end{itemize}
  \end{itemize}
\end{definition}

Der hauptsächliche Unterschied deses Konzeptes der \emph{order preserving} zum
originalen besteht in der abgeschwächten Darstellung der \emph{groundedness}.
Die beinhaltet die Regeln in $\Gamma^{X}_{\Pi}$ (über Bedingung 0b) als auch jene
aus $\Pi \backslash \Gamma^{X}_{\Pi}$. Der Rest der Definition unterscheidet sich
nicht von Definition \ref{def:2}.\\
Beispielsweise wird das \emph{answer set} $\{a, b\}$ aus Beispiel
\ref{example:pi3} durch die $<^W$-preserving Regelsequenz
$\langle r_3, r_2 \rangle$ erzeugt. hierbei ist zu beachten, dass $r_1$ die
Bedingung 2c erfüllt, jedoch nicht 2a noch 2b.\\
Diese abgeschwächten Darstellung der \emph{groundedness} kann einfach in die
Umwandlung von Definition \ref{def:3} integriert werden.

\begin{definition}
  Given the same prerequisites as in definition \ref{def:3}.
  Then, the logic program $\mathcal{T}^W(\Pi)$ over $L^+$ is defined as\\
  $\mathcal{T}^W(\Pi) = \bigcup_{r \in \Pi}\tau(r) \cup \{c_5(r, r')~|~r, r' \in \Pi\}$
  , where
    \begin{align*}
      c_5(r, r'): && ok'(n_r, n_{r'}) \hspace*{0.2cm}&\leftarrow \hspace*{0.2cm} (n_r \prec n_{r'}), head(r')
    \end{align*}
\end{definition}

Der Sinn von $c_5(r, r')$ ist es Regeln des Präferenz Prozesses auszuschließen,
nachdem der Kopf hinzugefügt wurde.


%\begin{proposition}
%Insert proposition here
%\end{proposition}
%\begin{proof}
%Insert proof here
%\end{proof}
%Text\\ %TODO
%Text\cite{reference1}

%The bibliography, done here without a bib file
%This is the old BibTeX style for use with llncs.cls

%Alternative bibliography styles:
%the following does the same as above except with alphabetic sorting
%\bibliographystyle{splncs_srt}
%the following is the current LNCS BibTex with alphabetic sorting
%\bibliographystyle{splncs03}
%If you want to use a different BibTex style include [oribibl] in the document class line
\flushleft
\bibliographystyle{splncs}
\bibliography{bibliography}

%\begin{thebibliography}{1}
%add each reference in here like this:
%\bibitem[RE1]{reference1}
%Author:
%Article/Book:
%Other info: (date) page numbers.
%\end{thebibliography}

\end{document}

