\section{Compiling order preservation}

Eine Übersetzung eines geordneten logischen Programmes $(\Pi, <)$ in ein
Standard logisches Programm $\Pi'$ wurde in \cite{delgrande2000logic} entwickelt.
Dieser Vorgang stellt sicher, das die aus $\Pi'$ resultierenden \emph{answer sets}
allen \emph{order preserving answer sets} von $\Pi$ entsprechen.

\begin{definition}
  Let $(\Pi, <)$ be a statically ordered program and let X be an answer set of $\Pi$.
  Then, X is called $<$-preserving, if there exists an enumeration
  $\langle r_i \rangle_{i \in I}$~of~$\Gamma_{\Pi}^X$ such that for every $i, j \in I$
  we have that:
  \begin{itemize}
    \item[0.] $body^+(r_i) \subseteq \{ head(r_j)~|~j<i\}$ and \\
    \item[1.] if $r_i < r_j$, then $j<i$ and \\
    \item[2.] if $r_i < r'$ and $r' \in \Pi \backslash \Gamma_{\Pi}^X$, then
      \begin{itemize}
        \item[(a)] $body^+(r') \not \subseteq X$ or
        \item[(b)] $body^-(r') \cap \{head(r_j)~|~j<i\} \not = \emptyset$
      \end{itemize}
  \end{itemize}
\end{definition}

Die Bedingung 0 macht die Eigenschaft der \emph{groundedness} explizit.\\
Bedingung 1 garantiert die Kompatibilität von $\langle r_i \rangle_{i \in I}$
zu $<$.\\
Bedingung 2 ist vergleichbar mit \emph{Condition II} aus Definition \ref{def:1}.
Durch sie können höherwertige Reglen nie durch niederwertige Blockiert werden.\\
Entsprechend \emph{Example} \ref{example:pi2} ist $X = \{p, b, \neg f, w\}$ das
einzige \emph{$<$-preserving} \emph{answer set} von $\Pi_2$. Es kann durch die 
\emph{grounded} Sequenzen
$\langle r_5, r_4, r_1, r_2 \rangle$ und $\langle r_5, r_1, r_4, r_2 \rangle$
erzeugt werden, welche beide die Bedingungen 1 und 2 erfüllen.
Die einzige \emph{grounded} Sequenz die $X' = \{p, b, f, w\}$ generiert,
$\langle r_5, r_4, r_2, r_3 \rangle$, verletzt Bedingung 2b.
Die entsprechende Übersetzung verbindet die Information der Ordnung
mit dem logischen Programm über das spezielle Prädikaten Symbol $\prec$.
Dies erlaubt ebenfalls eine Ordnung in einer Dynamischen Form.\\
Ein logisches Programm über einer propositionalen Sprache $L$ ist genau
dann dynamisch geordnet genannt, wenn $L$ die folgenden paarweisen disjunkten
Kategorien beinhaltet:
\begin{itemize}
  \item[(i)]   eine Menge $N$, welche die Namen der Regeln beinhaltet
  \item[(ii)]  eine Menge $At$ von Atomen eines Programmes
  \item[(iii)] eine Menge $At_{\prec}$ von präferenz Atomen $s \prec t$,
    wobei $s, t \in N$ Namen sind.
\end{itemize}
Für jedes Programm $\Pi$ existiert eine bijective Funktion $n(\cdot)$ welche jeder
Regel $r \in \Pi$ einen Namen $n(r) \in N$ zuweist.
Für eine vereinfachte Schreibweise wird $n_r$ anstatt $n(r)$ verwendet.
Gegebenen falls auch $n_i$ statt $n_{r_i}$.\\
Ein Atom $n_r \prec n_{r'} \in At_{\prec}$  lässt davon ausgehen, dass
$r < r'$ gilt.
Ein statisch geordnetes Programm $(\Pi, <)$ kann dem entsprechend durch ein
Programm, welches präferenz Atome nur anhand derer Fakten beinhaltet,
beschrieben werden. Die Schreibweise ist dann
$\Pi \cup \{(n_r \prec n_{r'}) \leftarrow~|~r < r'\}$.\\
Bei gegebenem $r < r'$ muss sichergestellt sein, das $r'$ vor $r$ berücksichtigt
wird. In dem Sinne, das bei einem \emph{answer set} X vor dem anwenden der
Regel $r$, zuerst die Regel $r'$ angewandt oder wiederlegt worden ist.
Dies wird durch das umformen der Regeln erziehlt, so das die Reihenfolge
der Anwendung der Regeln explizit kontrolliert werden kann.
Dadurch verfällt die Notwendigkeit vor dem anwenden von Regeln zu prüfen,
ob eine andere Regel vorher angewendet werden muss oder ungültig wird.
Für eine Regel $r$ existieren zwei Möglichkeiten das sie nich angewendet werden
müssen:
\begin{itemize}
  \item ein Literal aus $head^+(r)$ kommt nicht im \emph{answer set} vor, oder
  \item ein Literal aus $head^-(r)$ ist im \emph{answer set}
\end{itemize}
Zum ausdrücken einer Blockierung für jeder Regel $r$ in dem gegebenen Programm $\Pi$
wird ein neues Atom $bl(n_r)$ eingeführt.
Vergleichsweise dazu wird das Atom $ap(n_r)$ eingeführt, um eine mögliche anwendung
der Regel auszudrücken.
Zur Kontrolle der Reihenfolge der Anwendung wird das Atom $ok(n_r)$ eingeführt.
Unformell heißt dies das es dann \emph{ok} ist eine Regel $r$ anzuwenden, wenn
es \emph{ok} ist in bezug auf jede höherwertige Regel $r'$.
Dies ist dann der Fall, wenn die höherwertige Regel $r'$ blockiert oder angewandt
wurde.\\
Formal betrachtet, bei einem gegebenen dynamisch geordneten Programm $\Pi$ über
$L$, dann sei $L^+$ eine Sprache erreicht durch hinzufügen von, für alle
$r, r' \in \Pi$, neuen paarweisen disjunkten propositionalen Atomen
$ap(n_r)$, $bl(n_r)$, $ok(n_r)$ und $ok'(n_r, n_{r'})$.
Dann stellt die Übersetzung $\Gamma$ ein geordnetes Programm $\Pi$ über $L$ in
ein Standard Programm $\Gamma(\Pi)$ über $L^+$ wie folgt dar.

\begin{definition}
  Let $\Pi = \{r_1, ..., r_k\}$ be a dynamically ordered logic program over $L$.
  Then, the logic program $\mathcal{T}(\Pi)$ over $L^+$ is defined as
  $\mathcal{T}(\Pi) = \bigcup_{r \in \Pi}\tau(r)$, where $\tau(r)$ consists of the
  following rules, for $L^+ \in body^+(r)$, $L^- \in body^-(r)$, and $r'$,
  $r'' \in \Pi:$\\
  \begin{minipage}{0.8\textwidth}
    \begin{align*}
      a_1(r): && head(r) \hspace*{0.2cm}&\leftarrow \hspace*{0.2cm} ap(n_r)\\
      a_2(r): && ap(n_r) \hspace*{0.2cm}&\leftarrow \hspace*{0.2cm} ok(n_r), body(r)\\
      b_1(r, L^+): && bl(n_r) \hspace*{0.2cm}&\leftarrow \hspace*{0.2cm} ok(n_r), not~L^+\\
      b_2(r, L^-): && bl(n_r) \hspace*{0.2cm}&\leftarrow \hspace*{0.2cm} ok(n_r), L^-\\
      \\
      c_1(r): && ok(n_r) \hspace*{0.2cm}&\leftarrow \hspace*{0.2cm} ok'(n_r, n_{r_1}), ..., ok'(n_r, n_{r_k})\\
      c_2(r, r'): && ok'(n_r, n_{r'}) \hspace*{0.2cm}&\leftarrow \hspace*{0.2cm} not~(n_r \prec n_{r'})\\
      c_3(r, r'): && ok'(n_r, n_{r'}) \hspace*{0.2cm}&\leftarrow \hspace*{0.2cm} (n_r \prec n_{r'}), ap(n_{r'})\\
      c_4(r, r'): && ok'(n_r, n_{r'}) \hspace*{0.2cm}&\leftarrow \hspace*{0.2cm} (n_r \prec n_{r'}), bl(n_{r'})\\
      \\
      t(r, r', r''): && n_r \prec n_{r''} \hspace*{0.2cm}&\leftarrow \hspace*{0.2cm} n_r \prec n_{r'}, n_{r'} \prec n_{r''}\\
      as(r, r'): && \neg (n_{r'} \prec n_{r}) \hspace*{0.2cm}&\leftarrow \hspace*{0.2cm} n_r \prec n_{r'}
    \end{align*}
  \end{minipage}
\end{definition}

Sollte $\Pi'$ ein dynamisch geordnetes Programm sein welches $(\Pi, <)$ umfasst,
so wird $\Gamma(\Pi, <)$ als Schreibweise gegenüber $\Gamma(\Pi')$ vorgezogen.\\
Die ersten vier Regeln von $\tau(r)$ stellen die Anwendungen und Blockierungen
der originalen Regeln dar.\\
Die zweite Gruppe an Regeln bildet die Präferenz verarbeitung. Hierbei
quantifiziert $c_1(r)$ über die Regeln in $\Pi$. Dies ist notwendig bei der
Handhabung von dynamischen Präferenzen, denn diese können stark von dem
dazugehörigen Answer Set abhängen.
Die drei Regeln $c_2(r, r')$, $c_3(r, r')$ und $c_4(r, r')$ spezifizieren die
paarweise Abhängigkeit von Regeln in betracht auf die gegebene
Präferenzordnung. Für jedes Paar an Regeln $r, r'$ mit $n_r \prec n_{r'}$ wird
$ok'(n_r, n_{r'})$ erreicht, immer dann wenn $n_r \prec n_{r'}$ nicht hält oder
wenn entweder $ap(n_{r'})$ oder $bl(n_{r'})$ war ist. Dies erlaubt es $ok(n_r)$
abzuleiten, hinweisend darauf das $r$ möglicherweise anwendbar ist, immer dann
wenn für alle $r'$ mit $n_r \prec n_{r'}$, $r'$ schon angewendet wurde oder
nicht anwendbar ist.\\
Es ist wichtig zu vermerken, dass dies nur eine von vielen Möglichkeiten ist
um mit Präferenzen umzugehen. Andere Herangehensweisen können durch das ändern
von $ok(\cdot)$ und $ok'(\cdot, \cdot)$ erziehlt werden.


