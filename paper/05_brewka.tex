\section{Brewka and Eiter's concept of preference}


Ein weiterer Ansatz wurde von \cite{BrewkaEiter1999} vorgestellt. Wir nennen ein solches Answer-Set $B-preferred$. Dieser Ansatz unterschiedet sich erheblich an zwei verschiedenen Stellen von den beiden erstgenannten Ansätzen. Erstens das erstellen des Answer-Sets wird gesondert vom überprüfen der Bevorzugung betrachtet.
Zweitens werden Regeln, die zu mutmaßlich kontraintuitiven Ergebnissen führen würden, ausdrücklich vom Inferenzprozeß entfernt. Dies wurde explizit in \cite{BrewkaEiter2000}, durch den folgenden Filter, definiert. \\

  \begin{equation}
	Z_x(\Pi)~=~\Pi~\textbackslash~\{~r~\in~\Pi~|~head(r)~\in~X,~body^-(r)~\cap~X~\neq~\emptyset \}
  \label{eq:1}
\end{equation}

Als nächstes wird definiert das jedes $B-preferred$ Answer-Set von $\Pi$ auch ein Answer-Set von $Z_X(\Pi)$ ist. Jetzt wollen wir eine fixpoint-Charakterisierung der $B-preference$ geben.\\

\begin{definition}
  Let $(\Pi, <)$ be an ordered logic program and let $X$ be a set of literals.
  We define:
  \begin{equation*}
    \begin{split}
      X_0 & = \emptyset\hspace*{0.3cm} \text{~and for~} i \geq 0 \\
      X_{i+1} & = X_i \cup \{ head(r)~| \\
        & \hspace*{0.4cm}\left. \begin{aligned}
          I. &\hspace*{0.2cm} r \in \Pi \text{~is active wrt~} (X, X) \text{~and} \\
          II. &\hspace*{0.2cm} \text{there is no rule~} r' \in \Pi \text{~with~} r < r'
          \text{~such that} \\
          & \hspace*{0.2cm} \begin{aligned}
            (a) &\hspace*{0.2cm} r' \text{~is active wrt~} (X, X_i) \text{~and} \\
            (b) &\hspace*{0.2cm} head(r') \notin X_i
          \end{aligned}
        \end{aligned}
      \right \}
    \end{split}
  \end{equation*}
  Then, $C^{B}_{(\Pi, <)}(X) = \bigcup_{i\geq 0} X_i$ if $\bigcup_{i\geq 0} X_i$ is
  consistent. \\Otherwise, $C_{(\Pi, <)}(X) = Lit$.
  \label{def:1}
\end{definition}

Der Unterschied dieser Definition und der vorangegangen Definitionen manifestieren sich im Punkt \glqq I.\grqq. Hier wird die Aktivität (activ) auf $(X,X)$ anstatt auf $(X_i,X)$ überprüft. Durch dieses Kriterium sind im Programm ($\Pi_2$) jetzt beide Answer-Sets $B-prefered$ aber $\{p,~b,~\neg f,~w\}$ ist das einzige $W-$ und $D-preferred$ Answer-Set.\

Um dies etwas besser darzustellen, zeigen wir das die $B-preference$ auch in der order preservation Notation dargestellt werden kann. Diese würde der folgenden Definition entsprechen.\\

\begin{definition}
  Let $(\Pi, <)$ be a statically ordered lprogram and let X be an answer set of $\Pi$.
  Then, X is called $<^B$-preserving, if there exists an enumeration
  $\langle r_i \rangle_{i \in I}$~of~$\Gamma_{\Pi}^X$ such that, for every $i, j \in I$
  we have that:
  \begin{itemize}
		\item[1.] if $r_i < r_j$, then $j<i$ and \\
    \item[2.] if $r_i < r'$ and $r' \in \Pi \backslash \Gamma_{\Pi}^X$, then
      \begin{itemize}
        \item[(a)] $body^+(r') \not \subseteq X$ or
        \item[(b)] $body^-(r') \cap \{head(r_j)~|~j<i\} \not = \emptyset$
				\item[(c)] $head(r') \in X.$
      \end{itemize}
  \end{itemize}
\end{definition}

Diese Definition unterscheidet sich in zwei Punkten von seinen Vorangegangen.
Der erste Punkt ist das Fehlen des Punkte \glqq 0.\grqq.
Dieser würde die \glqq groundedness\grqq~ beschreiben.
Dies zeigt sich auch in der vorangegangen Definition an der Benutzung von $(X,X)$.
Der zweite Unterschied ist die Eigenschaft \glqq 2.(c)\grqq~, die
hinzugekommen ist. Diese stellt die Filterung der Elemente in \eqref{eq:1} dar.
