\section{Einleitung}

\begin{frame}
  Betrachtung von 3 Herangehensweisen f"ur ``Logik Programme mit Präferenzen''
  \begin{itemize}
    \item preferred alternating fixpoints
    \item compiling order preservation
    \item translations into standard logic programs
  \end{itemize}
\end{frame}


\begin{frame}{Logisches Programm}
  Ein ``general logic program'' ist eine Endliche Ansamlung von Regeln in der Form:
  \begin{figure}
    \begin{math}
      A_0~\leftarrow~A_1,~...,~A_m,~not~A_{m+1},~...,~not~A_n
    \end{math}
  \end{figure}
  bei dem gilt $n~\geq~m~\geq~0$\\[0.5cm]
  und jedes $A_i$ ist ein Atom.
\end{frame}


\begin{frame}{Erweitertes Logisches Programm}
  Ein ``extended logic program'' ist eine Endliche Ansamlung von Regeln in der Form:
  \begin{figure}
    \begin{math}
      L_0~\leftarrow~L_1,~...,~L_m,~not~L_{m+1},~...,~not~L_n
    \end{math}
  \end{figure}
  bei dem gilt $n~\geq~m~\geq~0$\\[0.5cm]
  und jedes $L_i$ ist ein Literal ($A$ oder $\neg A$).
\end{frame}


\begin{frame}{Negationen}
  schwache Negation:
  \begin{itemize}
    \item $cross~\leftarrow~not~train$
  \end{itemize}
  starke Negation:
  \begin{itemize}
    \item $cross~\leftarrow~\neg~train$
  \end{itemize}
\end{frame}


\begin{frame}{Definitionen}
  \begin{block}{\textbf{Define:} Lit}
    Lit ist die Menge aller Literale des Programms $\Pi$.
  \end{block}
  \begin{block}{\textbf{Define:}~Regel}
    $r = L_0~\leftarrow~L_1,~...,~L_m,~not~L_{m+1},~...,~not~L_n$
  \end{block}
  \begin{block}{\textbf{Define:}~$head$}
    $head(r) = L_0$
  \end{block}
  \begin{block}{\textbf{Define:}~$body$}
    $body(r) = {L_1,~...,~L_m,~not~L_{m+1},~...,~not~L_n}$
  \end{block}
\end{frame}


\begin{frame}{Definitionen}
  \begin{block}{\textbf{Define:}~$body^+$}
    $body^+(r) = {L_1,~...,~L_m}$
  \end{block}
  \begin{block}{\textbf{Define:}~$body^-$}
    $body^-(r) = {not~L_{m+1},~...,~not~L_n}$
  \end{block}
  \begin{block}{\textbf{Define:} $basic$}
    Ein Programm wird $basic$ genannt, wenn f"ur alle Regeln gilt:\\
    $body^-(r)~=~\emptyset$
  \end{block}
  \begin{block}{\textbf{Define:} $ground$}
    Eine Regel $r$ wird $ground$ genannt, wenn keine Variablen darin enthalten sind\\
  \end{block}
\end{frame}


\begin{frame}{Definitionen}
  \begin{block}{\textbf{Define:} $reduct$ von r}
    $r^+~=~head(r)~\rightarrow~body^+(r)$
  \end{block}
  \begin{block}{\textbf{Define:} $reduct$ von $\Pi$}
    $\Pi$ in relation zu $X$. $X$ ist eine Menge von Literalen\\
    $\Pi^X~=\{r^+~|~r \in \Pi~and~body^-(r) \cap X = \emptyset \}$
  \end{block}
\end{frame}


\begin{frame}{Definitionen}
  \begin{block}{\textbf{Define:} $logically~closed$}
    We say that $X$ is logically closed iff it is either consistent or equals Lit.
  \end{block}
  \begin{block}{\textbf{Define:} $closed~under~a~basic~program$}
    A set of literals $X$ is closed under a basic program $\Pi$ iff for
    any $r~\in~\Pi$, $head(r)~\in~X$ whenever $body^+(r) \subseteq X$.
  \end{block}
\end{frame}


\begin{frame}{Definitionen}
  \begin{block}{\textbf{Define:} Operator $Cn(\Pi)$}
  $Cn(\Pi)$ bezeichnet die kleinste Menge von Literalen die sowohl
  $logically~closed$, als auch $closed~under~a~basic~program~\Pi$ ist
  \end{block}
  \begin{block}{\textbf{Define:} $answer~set$}
    $Cn(\Pi^X)=X$
  \end{block}
\end{frame}


