

\begin{frame}
  Betrachtung von 3 Herangehensweisen f"ur Logik Programme mit Präferenzen
  \begin{itemize}
    \item Preferred alternating fixpoints
    \item Compiling order preservation
    \item translations into standard logic programs
  \end{itemize}
\end{frame}


\begin{frame}{Logisches Programm}
  Ein Logisches Programm ist eine Endliche Ansamlung von Regeln in der Form:
  \begin{figure}
    \begin{math}
      L_0~\leftarrow~L_1,~...,~L_m,~not~L_{m+1},~...,~not~L_n
    \end{math}
  \end{figure}
  bei dem gilt $n~\geq~m~\geq~0$\\[0.5cm]
  und jedes $L_i (0 \leq i \leq n)$ ist ein Literal also ein Atom $A$ oder seine Negation $\neg A$.\\
\end{frame}

\begin{frame}{Definitionen}
  \begin{block}{\textbf{Define:} Lit}
    Lit ist die Menge aller Literale.
  \end{block}
  \begin{block}{\textbf{Define:}~$head$}
  $head(r) = L_0$
  \end{block}
  \begin{block}{\textbf{Define:}~$body$}
  $body(r) = {L_1,~...,~L_m,~not~L_{m+1},~...,~not~L_n}$
  \end{block}
\end{frame}


\begin{frame}{Definitionen}
  \begin{block}{\textbf{Define:}~$body^+$}
  $body^+(r) = {L_1,~...,~L_m}$
  \end{block}
  \begin{block}{\textbf{Define:}~$body^-$}
  $body^-(r) = {not~L_{m+1},~...,~not~L_n}$
  \end{block}
\end{frame}

\begin{frame}{Definitionen}
  \begin{block}{\textbf{Define:} basic}
    Ein Programm wird basic genannt, wenn f"ur alle Regeln gilt:\\
    $body^-(r)~=~\emptyset$
  \end{block}
  \begin{block}{\textbf{Define:} reduct of rule r}
    $r^+~=~head(r)~\rightarrow~body^+(r)$
  \end{block}
  \begin{block}{\textbf{Define:} reduct of program $\Pi$ to a set X of literals}
    $\Pi^X~=\{r^+~|~r \in \Pi~and~body^-(r) \cap X = \emptyset \}$
  \end{block}
\end{frame}


\begin{frame}{Definitionen}
  \begin{block}{\textbf{Define:} }
    $body^-(r)~=~\emptyset$
  \end{block}
\end{frame}
%A set of literals X is closed under a basic program Π iff for
%any r ∈ Π, head (r) ∈ X whenever body + (r) ⊆ X. We say
%that X is logically closed iff it is either consistent (ie. it does
%not contain both a literal A and its negation ¬A) or equals
%Lit. The smallest set of literals which is both logically closed
%and closed under a basic program Π is denoted by Cn






Our investigation adopts characterization techniques found in
the same literature in order to shed light on the relationships
among these approaches. This provides us with different
characterizations in terms of (i) fixpoints, (ii) order preser-
vation, and (iii) translations into standard logic programs.

While the two former provide semantics for logic program-
ming with preference information, the latter furnishes im-
plementation techniques for these approaches.

(i) may
be regarded as a meta-level description of the corresponding
construction process.

One may view (ii) as the most seman-
tical characterization because it tells us which “models” of
the original program are selected by the respective preference
handling strategy.


